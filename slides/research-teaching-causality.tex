\documentclass[aspectratio=169, table]{beamer}

%\usepackage[beamertheme=./praditatheme]{Pradita}
\usepackage[utf8]{inputenc}
\usepackage{xcolor} % for color
\usepackage{colortbl} % for table color
\usepackage{multirow}

\usetheme{Pradita}
%
\subtitle{the 2025 4th International Conference on\\Electronics Representation and Algorithm (ICERA)}

\title{Deriving IT Strategy from the Research-\\Teaching Causality of Times Higher\\Education Rankings}
\date[Serial]{\scriptsize {STMIK El Rahma, Yogyakarta, 12 June 2025}}
\author[Pradita]{\footnotesize{\textbf{Alfa Yohannis, Alexander Waworuntu, Januponsa Dio Firzqi}}}

\begin{document}

\frame{\titlepage}

\begin{frame}[fragile]
\frametitle{Contents}
\vspace{20pt}
\begin{columns}[t]
\column{0.5\textwidth}
\tableofcontents[sections={1-4}]

\column{0.5\textwidth}
\tableofcontents[sections={5-10}]
\end{columns}
\end{frame}

\section{Introduction}

\begin{frame}[fragile]{Research Motivation}
	\vspace{20pt}
	
	\begin{itemize}
		\item Universities aim to excel in both \textbf{teaching} and \textbf{research}.
		\item Their relationship is debated: synergy or conflict?
		\item Existing policies often treat them as separate domains.
		\item There is a need for \textbf{evidence-based strategies} to align both effectively.
	\end{itemize}
\end{frame}

\begin{frame}[fragile]{Research Questions}
	\vspace{20pt}
	\begin{itemize}
		\item Do teaching and research influence each other over time?
		\item What causal patterns exist across institutions?
		\item Can IT strategies help align and strengthen both domains?
	\end{itemize}
\end{frame}

\section{Background}

\begin{frame}[fragile]{Related Work Summary}
	\vspace{20pt}
	\begin{itemize}
		\item Past findings vary: conflict (Maisano et al., 2023) vs. synergy (Himanka, 2024).
		\item Most studies focus on correlation, not causality.
		\item Post-pandemic calls for integration (Hordósy \& McLean, 2022).
		\item Lack of \textbf{actionable strategies} derived from empirical analysis.
	\end{itemize}
\end{frame}

\section{Methodology}

\begin{frame}[fragile]{Methodology}
	\vspace{20pt}
	\begin{itemize}
		\item \textbf{Dataset}: Times Higher Education (THE) Rankings, 2011–2023
		\\variables---overall score, rank,
		research score, teaching score, citations score, intl students,
		international outlook score, student staff ratio, industry income
		score, number students, and year.
		\item \textbf{Sample}: 346 universities with $\geq$ 10 years of data.
		\item \textbf{Method}: Granger Causality Test with Lag = 1 and 2.
		\item \textbf{Categories}:
		\begin{itemize}
			\item RT: Research $\rightarrow$ Teaching
			\item TR: Teaching $\rightarrow$ Research
			\item Bi: Bidirectional
			\item None: No causality
		\end{itemize}
	\end{itemize}
\end{frame}

\section{Results and Analysis}


\begin{frame}[fragile]{\Large{Granger Causality Summary (L = 1, L = 2, and Combined)}}
	\vspace{20pt}
	\arrayrulecolor{black}
	\setlength{\arrayrulewidth}{0.3pt}
	\begin{center}
		\begin{tabular}{|l|r|r|r|r|r|r|}
			\hline
			\multirow{2}{*}{\textbf{Category}} 
			& \multicolumn{2}{c|}{$L=1$} 
			& \multicolumn{2}{c|}{$L=2$} 
			& \multicolumn{2}{c|}{Combined} \\ \cline{2-7}
			& \textbf{Count} & \textbf{\%}
			& \textbf{Count} & \textbf{\%}
			& \textbf{Count} & \textbf{\%} \\ \hline
			\textbf{RT}     & 27  & 7.8\%  & 18  & 5.2\%  & 41  & 11.8\% \\ \hline
			\textbf{TR}     & 37  & 10.7\% & 31  & 9.0\%  & 54  & 15.6\% \\ \hline
			\textbf{Bi}     & 11  & 3.2\%  & 11  & 3.2\%  & 20  & 5.8\%  \\ \hline
			\textbf{None}   & 271 & 78.3\% & 286 & 82.7\% & 231 & 66.8\% \\ \hline
		\end{tabular}
	\end{center}
	
	\vspace{10pt}
	Summary of Granger causality classifications across $L=1$, $L=2$, and their union ($p$-value $\leq 0.05$, $n=346$). RT: Research $\rightarrow$ Teaching, TR: Teaching $\rightarrow$ Research, Bi: Bidirectional, None: No Causality.
\end{frame}


\begin{frame}[fragile]{Average Rank and Score by Causality Category}
	\vspace{20pt}
	\arrayrulecolor{black}
	\setlength{\arrayrulewidth}{0.3pt}
	\begin{center}
		\begin{tabular}{|l|r|r|r|r|r||r|r|r|r|r|}
			\hline
			\multirow{2}{*}{\textbf{Cate}} 
			& \multicolumn{5}{c||}{\textbf{Rank}} 
			& \multicolumn{5}{c|}{\textbf{Score}} \\ \cline{2-11}
			& \textbf{Avg} & \textbf{Std} & \textbf{Med} & \textbf{Min} & \textbf{Max}
			& \textbf{Avg} & \textbf{Std} & \textbf{Med} & \textbf{Min} & \textbf{Max} \\ \hline
			\textbf{RT}   & 261.3 & 163.1 & 236.5 & 5.1   & 820.9 & 51.1 & 13.1 & 48.0 & 37.2 & 93.5 \\ \hline
			\textbf{TR}   & 242.8 & 155.5 & 223.2 & 2.4   & 678.4 & 52.7 & 14.0 & 48.8 & 37.2 & 94.6 \\ \hline
			\textbf{Bi}   & 175.6 & 123.1 & 135.6 & 14.5  & 498.9 & 56.8 & 12.3 & 55.5 & 41.8 & 86.0 \\ \hline
			\textbf{None} & 242.7 & 169.6 & 207.4 & 2.2   & 749.1 & 52.8 & 12.6 & 50.1 & 37.4 & 94.5 \\ \hline
		\end{tabular}
	\end{center}
	
	\vspace{8pt}
	Summary of Combined Causality Results for average institutional rank and overall score ($p$-value $\leq 0.05$, $n=346$). RT: Research $\rightarrow$ Teaching, TR: Teaching $\rightarrow$ Research, Bi: Bidirectional, None: No Causality.
\end{frame}




\begin{frame}[fragile]{Mann-Whitney U Test for Ranks and Scores}
	\vspace{20pt}
	
	The Mann-Whitney U test shows significant differences in rank and score across causality types:
	
	\begin{itemize}
		\item \textbf{Bidirectional (Bi)} institutions rank and score higher than both \textbf{RT} ($p \leq 0.05$) and \textbf{TR} ($p \leq 0.10$).
		\item \textbf{RT} and \textbf{TR} perform significantly worse than \textbf{Bi}, with TR showing slightly weaker significance.
		\item \textbf{None} also ranks and scores lower than \textbf{Bi} ($p \leq 0.10$).
		\item No significant differences were found between \textbf{RT}, \textbf{TR}, and \textbf{None} among themselves.
	\end{itemize}
	
	Institutions with bidirectional causality between teaching and research tend to perform better overall.
\end{frame}





%\begin{frame}[fragile]{Significance Testing}
%	\vspace{20pt}
%	\begin{itemize}
%		\item Mann-Whitney U test confirms performance ranking:
%		\begin{itemize}
%			\item Bidirectional $>$ TR $>$ RT $>$ None
%		\end{itemize}
%		\item F-statistics strongest in bidirectional institutions.
%		\item Stronger causality corresponds with higher performance.
%	\end{itemize}
%\end{frame}

\section{Discussion and IT Strategy Recommendations}

\begin{frame}[fragile]{Strategic Implications and IT Strategies}
	\vspace{20pt}
	\begin{columns}[t]
		\column{0.48\textwidth}
		\textbf{Strategic Implications}
		\begin{itemize}
			\item Most universities separate teaching and research.
			\item Bidirectional cases show \textbf{mutual reinforcement}.
			\item Institutions should realign policy from separation to \textbf{strategic integration}.
		\end{itemize}
		
		\column{0.48\textwidth}
		\textbf{Proposed IT Strategies}
		\begin{enumerate}
			\item Unified LMS + Research Platforms
			\item Collaborative tools for faculty interaction
			\item Personalised student pathways linked to research
			\item Data-driven dashboards for planning
			\item Scalable cloud-based infrastructure
			\item Faculty training on IT tools
			\item Cross-disciplinary research-teaching networks
		\end{enumerate}
	\end{columns}
\end{frame}


\section{Conclusion}

\begin{frame}[fragile]{Conclusion}
	\vspace{20pt}
	\begin{itemize}
		\item Most universities show \textbf{no causal link} between teaching and research.
		\item \textbf{Teaching $\rightarrow$ Research} more frequent than the reverse.
		\item Bidirectional cases have the \textbf{best scores and ranks}.
		\item \textbf{IT strategies} can bridge and enhance both domains.
		\item Institutions should \textbf{invest in integration via technology}.
	\end{itemize}
\end{frame}
%
%
%%\section{Pola Perilaku Lanjutan}
%%
%%\section{Pola Mediator}
%%\begin{frame}{\hfill}
%%	\centering
%%	\textbf{\Huge{Pola Mediator}}
%%\end{frame}



\end{document}
